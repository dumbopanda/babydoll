\chapter{Introduction}
%\pagenumbering{arabic}
\hspace{3mm} The global transition towards sustainable transportation has catalyzed the rapid adoption of electric vehicles (EVs). Within this domain, Electric Two-Wheelers (E2Ws), encompassing e-scooters and e-bikes, have emerged as a cornerstone of modern urban mobility. Their proliferation is driven by low operational costs, convenience in congested urban environments, and a decreasing environmental footprint. However, this technological shift is accompanied by a significant and highly publicized setle of challenges, the most critical of which is the thermal management of their high-energy-density Lithium-ion (Li-ion) battery packs.

This report is submitted as part of the project "Development of a Hybrid Battery Thermal Management System for Electric Two Wheelers." The overarching goal of this project is to design, simulate, and optimize a novel thermal management solution tailored to the unique constraints of E2W applications. However, the design of any effective thermal management system is fundamentally predicated on a precise, predictive understanding of the heat source it must control.

Therefore, the specific focus of this report is the foundational prerequisite for that design: the development, parameterization, and validation of a high-fidelity, computationally efficient electro-thermal model for the battery cell. This work will utilize the Equivalent Circuit Model (ECM) approach, coupled with an energy balance model, to create a "Thermal ECM." This model will serve as the primary engineering tool, providing the critical heat generation data required to intelligently design, size, and optimize the target hybrid Battery Thermal Management System (BTMS).

\section{Context and Problem Definition}
The need for effective thermal management arises from the immediate danger of thermal runaway and the long-term risk of deteriorating battery performance and life. The E2W application, in particular, is a rare combination of high risk and demanding engineering constraints.

The most acute driver for BTMS is safety. The number of fire incidents in personal mobility devices has increased drastically and therefore, public and regulatory concern is justifiable. All these events mainly involve thermal runaway of Li-ion battery packs \cite{FSRI_Escooter_FireHazards_2025}. In other words, the failure of battery packs leads to events presented in this paper. Thermal runaway describes an uncontrolled self-heating exothermic reaction cascade within a cell. It results in high temperatures, flammable gases, and a risk of fire or explosion. The risk in electric two-wheelers is exacerbated by their small size. The design has high energy density cells that are placed too closely together. This means the failure of one cell can quickly spread to the entire pack. Studies showed that the flashover of a room can be reached in a matter of seconds for an E2W battery fire and conditions would be “immediately fatal” to the occupants \cite{FSRI_Escooter_FireHazards_2025}.

For a vehicle to perform reliably and economically, a battery thermal management system (BTMS) is necessary. The chemical processes in a Li-ion battery are very sensitive to temperature \cite{Chargie_BatteryDegradation_2024}. This sensitivity defines a narrow optimal operating window. The internal resistance of a battery usually becomes very high at low temperatures. When AC voltage gets interrupted, it cuts off the power so you don’t get acceleration. You also have reduced power acceptance. This makes regenerative braking and charging basically useless.

The method that causes degradation is a measurable physical phenomenon and not just a concept. Liaw et al.'s early work on modeling battery life showed that the capacity fade resulting from thermal aging and harsh cycling is directly reflected in the parameters of an Equivalent Circuit Model \cite{LiawModeling}. In other words, they showed that degradation takes place as a measurable increase in polarization resistance of the cell (denoted as ‘R2’) over the life of the cell (see Figure 4.5). We create an important link here: to have an effective BTMS is in effect a life extension program. Keeping a cool and steady temperature, a BTMS directly inhibit the parasitic reactions that cause this internal resistance to increase, thus maximally extending the usable service life of the battery and protecting the large financial investment it represents.

\section{Modeling as a Foundation for BTMS Design}
To tackle safety and life issues, parent project proposed ‘Hybrid’ BTMS. A review of existing thermal management strategies pointed out that they appear three-fold active, passive and hybrid Active systems provide significantly higher cooling performance as compared to passive systems. An industry standard in high-performance electric cars. Yet, with their high cost and complexity, weight as well as parasitic power draw from pumps and radiators, they are generally unsuitable for the cost- and space-sensitive E2W \cite{BatteryDesign_BTMS_2024}. Passive systems use PCM or Phase Change Materials. They are very simple and silent. They do not consume any power as well. They absorb heat in the form of latent heat during their phase change. The main disadvantage of these materials is that they have a limited amount of thermal energy they can hold, that is once the PCM is completely melted, it cannot function anymore. Moreover, due to low thermal conductivity, their rate of releasing thermal energy is quite slow.

This, however, leads to the central problem of this report. What is the peak thermal load (Watt) during the acceleration? This is a critical question for the engineer designing such a hybrid system. What is the average sustained heat load? How much PCM mass is required to absorb the peaks? What amount of fan airflow, expressed in CFM, is necessary to reject the average load? In order to answer these questions, a predictive dynamic model of the heat generation of the battery is required. Without this model, the BTMS design is mere guesswork.

This report intends to contribute the necessary predictive tool for this purpose. There are very complex electrochemical models (e.g., Pseudo-two-Dimensional, P2D) available; however, due to their high cost these cannot be used for implementing real-time control. Because of this reason, the chosen method is Equivalent Circuit Model (ECM). The method represents the battery dynamic behaviour using a combination of voltage sources, resistors and capacitors. ECMs are computationally efficient and rely on simple Ordinary Differential Equations (ODEs) which makes them the industry standard for the real-time estimation within a BMS.
This report focuses on the development of a "Thermal ECM," which is an electrical ECM and a thermal energy balance model. This method is powerful because it maps the ECM parameters directly to the two main sources of dissipation: The ‘irreversible’ (joule) heat (Qirr), or heat from the internal resistances of the cell (overpotential). It is calculated using the ECM parameters as: Qirr=I*(Vt-OCV). Where I is the current, Vt is the terminal voltage, and OCV is the open-circuit voltage. Reversible heat is defined as the thermodynamic effect created in the cell from the heat of reaction and changes in entropy. Depending on the SOC and direction of current, it can either be positive (producing heat) or negative (absorbing heat). We can find the reversible capacity from OCV parameter as Qrev= - I*T*(dOCV/dT), where ‘T’ is the cell temperature and ‘(dOCV/dT)’ is the entropic heat coefficient. The complete development, parameterization, and validation of a 2nd order, coupled electro-thermal Equivalent Circuit Model of a high capacity Li-ion cell has been described in this report. The new model aims to overcome the restrictions of conventional ECMs by intelligent corrective measures for high C-rate operation and capacity fade as proposed by Cho et al\cite{ChoThermalECM}. The validated model formed will be the predictive tool to quantify (Qirr+Qrev). The subsequent data-driven design, sizing, and simulation will be for the target hybrid BTMS for the electric two-wheeler. 

\section{Organization of The Report} To explain the development and validation of electro-thermal battery model, the report is divided into six chapters.

In this \textbf{chapter 1}, introduction of research domain has been presented. Challenges of thermal in E2W applications have been discussed. Need of Battery Thermal Management System (BTMS) have been established. Role of Equivalent Circuit Modelling (ECM) has been laid down. This is the main focus of this thesis.

\textbf{Chapter 2} presents a comprehensive literature review. This review has been segregated into two parts. The first part gives a survey of hybrid BTMS technologies and their suitability to E2W constraints. The second part discusses battery modelling theories, their application and state-of-the-art for a thermal analysis. This justifies the choice of the Thermal ECM.

The aims of the research has been elaborated in\textbf{ chapter 3 }It specifies the exact technical, quantifiable goals of the model including its architecture, parameterization, ‘fixes’ to be made, and targeted validation accuracy.

\textbf{Chapter 4} consists of results and discussion of the present study. The last chapter presents the final ECM parameters as functions of SOC and temperature, followed by the validation plots that compare simulated voltage and temperature to experimental data, and a discussion of the importance of the high C-rate corrections.

At last, \textbf{chapter 5} provides an overview of the key findings mentioned in this report. In addition, it specifies vital next steps – how this validated ECM will be employed as a “digital twin” to simulate the E2W drive cycles and subsequently design, size and optimize the target hybrid BTMS.
