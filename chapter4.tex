\chapter{Methodology}

%\pagenumbering{arabic}\hspace{3mm}

\label{ch:results}



The methodology, which explained in the previous chapter, was used on a HPPC data set to extract the parameters for a second-order ECM. The results of the extraction process will be discussed here along with the designated parameters and major trends. This will also help create the final Simulink model. 



\section{Data Source and Extracted Parameters}



The experimental HPPC test data used for this project was taken from a MathWorks example as shown in Figure 4.1. The BAK N18650CL-29 cell is a lithium-ion cell of type 18650. The tests took place at five uniform ambient temperatures, specifically 0$^{\circ}$C, 10$^{\circ}$C, 25$^{\circ}$C, 35$^{\circ}$C and 45$^{\circ}$C, which enabled temperature dependent parameterization.



% --- FIGURE PLACEHOLDER 1 ---

% Replaced \includegraphics which was causing a "File not found" error.

% The LaTeX environment cannot access local files.

\begin{figure}[htbp] 
\centering \includegraphics[
                    width=0.9\textwidth,
                    height=0.4\textheight,
                    keepaspectratio
                    ]{figures/Data.png} \caption{Source and description of the HPPC test data used for parameter extraction (Image courtesy of MathWorks, Inc.)} \label{fig:4.1} 
\end{figure}



The methodology for parameter extraction was implemented at each temperature and SOC point present in the dataset. The ohmic resistance (R0) of the second-order ECM, two polarization resistances (R1, R2), and two polarization capacitances (C1, C2) parameters are plotted as a function of the State of Charge (SOC).



% --- FIGURE PLACEHOLDER 2 (SUBFIGURES) ---

\begin{figure}[htbp]
    \centering
    % Fits within 90% width AND 40% height, whichever is smaller
    \includegraphics[width=0.9\textwidth, 
                     height=0.4\textheight, 
                     keepaspectratio]{figures/Em.jpg}
    \caption{Open-Circuit Voltage (OCV) vs. SOC}
    \label{fig:4.2}
\end{figure}

\begin{figure}[htbp] 
\centering \includegraphics[width=0.9\textwidth]{figures/R0.jpg} \caption{Ohmic Resistance ($R_o$) vs. SOC} \label{fig:4.3} 
\end{figure}

\begin{figure}[htbp] 
\centering \includegraphics[width=0.9\textwidth]{figures/R1.jpg} \caption{Fast Polarization Resistance ($R_1$) vs. SOC} \label{fig:4.4} 
\end{figure}



% --- FIGURE PLACEHOLDER 3 (SUBFIGURES) ---

\begin{figure}[htbp] 
\centering \includegraphics[width=0.9\textwidth]{figures/R2.jpg} \caption{Slow Polarization Resistance ($R_2$) vs. SOC} \label{fig:4.5} 
\end{figure}
    \hfill

    % Placeholder for C2.png

  \begin{figure}[htbp] 
\centering \includegraphics[width=0.9\textwidth]{figures/C1.jpg} \caption{Fast Polarization Capacitance ($C_1$) vs. SOC} \label{fig:4.6} 
\end{figure}

  \begin{figure}[htbp] 
\centering \includegraphics[width=0.9\textwidth]{figures/C2.jpg} \caption{Slow Polarization Capacitance ($C_2$) vs. SOC} \label{fig:4.7} 
\end{figure}

\section{Analysis of Parameter Trends}



An in-depth study of the plotting reveals the battery’s internal parameters’ highly nonlinearity which is the essential aspect of the work\cite{LiawModeling,ChoThermalECM}.



\subsection{Ohmic Resistance ($R_o$)}

According to Fig. 4.3, the ohmic resistance R0 is within the milli-ohm range; in particular, it varies between about 2.55 m$\Omega$ and 2.75 m$\Omega$. While SOC is clearly used, the exhibit. Shows a regular sawtooth pattern – not a simple linear or quadratic function. This part is for the instantaneous voltage drop which plays an important part in capturing the instantaneous response of the model to a current pulse.



\subsection{Polarization Resistances ($R_1$ and $R_2$)}

The polarization resistances R1 and R2 correspond to the charge transfer and the mass transport limitations in the cell. Both charts show a very strong non-linear dependence on stratagem.



The primary conclusion drawn here is the steep and dramatic rise in both resistances at low state of charges (e.g. 20\% SOC). This is consistent with the known physics of batteries, which state that the electric potential to intercalate/de-intercalate lithium ions increases exponentially when the solid-state solid of the electrode has almost no lithium (such as during discharge). R1 and R2 also show complicated spiky behaviour for mid-SOC range indicating that these parameters cannot be represented by a simple mathematical equation.



\subsection{Polarization Capacitances ($C_1$ and $C_2$)}

C1 and C2 are the two capacitances which represent the capacitive effects of the double-layer and the diffusion processes.  These parameters show the most extreme non-linear behavior.



The “fast” capacitance, C1, exceeds 2000 Farads at select SOC points (e.g., 0.2, 0.4, 0.8), but virtually vanishes at others; see figure 4.6 placeholder. The capacitance denoted as C2, which is a slow charge capacitance, similarly behaves like that of C1, whereby its value becomes larger, reaching the thousands of Farads range (remember the factor of 10$^4$ on the Y-axis in the original plot).



The need for the HPPC test is highlighted by this erratic condition\cite{MathWorksHPPC}. The relationships between complex parameters which are captured in simple 2D lookup tables (Quality of Charge/SOC versus Temperature) make it possible for the Simulink model to predict the battery voltage response throughout the operating range.



\section{Conclusion}

The results of ECM parameter extraction were discussed in this chapter. The data were obtained from a documented experimental HPPC test performed on a BAK N18650CL-29 cell. The five second-order ECM parameters were extracted and plotted against the State-of-Charge ($R_o, R_1, C_1, R_2, C_2$).



The main finding is that the five parameters are highly non-linear and complex functions of SOC. Therefore, the use of a lookup table approach within the data-driven framework is justified, as opposed to a simple empirical equation. The resulting parameter sets obtained at five different temperatures are at the core of the battery model to be implemented and validated in Matlab Simulink.



The subsequent chapter will present the conclusion and some ideas for the future.